\documentclass[12 pt]{beamer}
\usetheme[
	bullet=circle,		% Other option: square
	bigpagenumber,		% circled page number on lower right
	topline=true,			% colored bar at the top of the frame 
	shadow=false,			% Shading for beamer blocks
	watermark=BG_lower,	% png file for the watermark
	]{Flip}


\newcommand{\titleimage}{title}			% Custom title 
\newcommand{\tanedo}{tanedolight}		% Custom author name
\newcommand{\CMSSMDM}{CMSSMDMlight.png}	% light background plot


%%%%%%%%%%
% FONTS %
%%%%%%%%%%

\usepackage[T1]{fontenc}
%\usepackage{lmodern}		
%\usepackage{sfmath}		% Sans Serif Math, off by default

%% Protects fonts from Beamer screwing with them
%% http://tex.stackexchange.com/questions/10488/force-computer-modern-in-math-mode
\usefonttheme{professionalfonts}


\usepackage[no-math]{fontspec}		

%\defaultfontfeatures{Mapping=tex-text}	% This seems to be important for mapping glyphs properly

\usepackage{amsmath}
%\usepackage{amsfonts}
%\usepackage{amssymb}
\usepackage{mathspec}
\usepackage{graphicx}
%\usepackage{mathrsfs} 			% For Weinberg-esque letters
\usepackage{cancel}				% For "SUSY-breaking" symbol
\usepackage{slashed}            % for slashed characters in math mode
%\usepackage{bbm}                % for \mathbbm{1} (unit matrix)
\usepackage{amsthm}				% For theorem environment
\usepackage{multirow}			% For multi row cells in table
\usepackage{arydshln} 			% For dashed lines in arrays and tables
\usepackage{tikzfeynman}		% For Feynman diagrams
% \usepackage{subfig}           % for sub figures
% \usepackage{young}			% For Young Tableaux
% \usepackage{xspace}			% For spacing after commands
% \usepackage{wrapfig}			% for Text wrap around figures
% \usepackage{framed}

\setsansfont{calibri}[ 
Extension = .ttf,
UprightFont = *,
BoldFont = *b,
ItalicFont = *i,
Scale = 1
]

\setmathfont(Digits,Latin,Greek){SitkaI.ttc}

\graphicspath{{images/}}	% Put all images in this directory. Avoids clutter.


\usetikzlibrary{backgrounds}
\usetikzlibrary{mindmap,trees}	% For mind map
% http://www.texample.net/tikz/examples/computer-science-mindmap/


% SOME COMMANDS THAT I FIND HANDY
% \renewcommand{\tilde}{\widetilde} % dinky tildes look silly, dosn't work with fontspec
\newcommand{\comment}[1]{\textcolor{comment}{\footnotesize{#1}\normalsize}} % comment mild
\newcommand{\Comment}[1]{\textcolor{Comment}{\footnotesize{#1}\normalsize}} % comment bold
\newcommand{\COMMENT}[1]{\textcolor{COMMENT}{\footnotesize{#1}\normalsize}} % comment crazy bold
\newcommand{\Alert}[1]{\textcolor{Alert}{#1}} % louder alert
\newcommand{\ALERT}[1]{\textcolor{ALERT}{#1}} % loudest alert
%% "\alert" is already a beamer pre-defined



\author{Gergely Vakulya and Helga Anna Albert-Huszár}
\title{Gamifying Cybersecurity:\\The CTF Challenges}
\institute{}
\date{}



\begin{document}

%% To use external nodes; http://www.texample.net/tikz/examples/beamer-arrows/
\tikzstyle{every picture}+=[remember picture]

{
\setbeamertemplate{sidebar right}{\llap{\includegraphics[width=\paperwidth,height=\paperheight]{backgnd}}}

\begin{frame}[c]
\begin{center}
	% \includegraphics[width=7cm]{WarpedPenguinsReturn}

  \Large
  \textbf{Gamifying Cybersecurity:}

  \textbf{The CTF Challenges}

  \qquad
  
  \textit{Gergely Vakulya\\and\\Helga Anna Albert-Huszár}
  
  \qquad

  AIS 2024
  

	%\begin{tikzpicture}%[show background grid] %% Use grid for positioning, then turn off
	%	\node[inner sep=0pt,above right] (title) 
	%		{ \includegraphics[width=7cm]{\titleimage} };
	%	% \node (title) at (1.5,1.5) {};
	%\end{tikzpicture}
	%\quad

	% \includegraphics[width=7cm]{\titleimage} 
	
	%\vspace{1em}
	%\footnotesize\textcolor{gray}{Journal of Cool Beans
	%\texttt{[arXiv:1234.5678]}}
	%\vspace{.5em}
	
	%\includegraphics[height=1.5cm]{\tanedo} \quad
	 % {\fontspec{Zapfino} Flip Tanedo} \quad
	% \includegraphics[height=1cm]{FlipSansSerif} \quad
	%\includegraphics[height=1.5cm]{CUasym}\\
	% \footnotesize\textcolor{gray}{In collaboration with} Csaba Cs\'aki\textcolor{gray}{,} Yuval Grossman\textcolor{gray}{, and} Yuhsin Tsai\normalsize\\
	%	\footnotesize\textcolor{gray}{In collaboration with 
%		D. Grayson, J. Todd, T. Drake, S. Brown, D. Wayne}\normalsize\\
%	\textcolor{normal text.fg!50!Comment}{\textit{Gotham University}, \today}
	% \textcolor{Comment}{ \;($\pi$ day)}\\
	% \Comment{4 February 2011}
\end{center}
\end{frame}
}

%------------------------------------------------

%\setbeamertemplate{default}{}

\begin{frame}{Cybersecurity education}

  \begin{block}{Main targets}
    \begin{itemize}
        \item{Secure software development}
        \item{Understanding network protocols}
    \end{itemize}
  \end{block}

  \begin{block}{Problems to be solved}
    \begin{itemize}
        \item{New generation: New education methods}
          \begin{itemize}
            \item{Gamification}
            \item{Challenges, learning by doing}
            \item{Immediate feedback and rewarding}
          \end{itemize}
        \item{}
    \end{itemize}
  \end{block}

\end{frame}

%------------------------------------------------

\begin{frame}{CTF: Capture The Flag}

  \begin{block}{Origins}
    \begin{itemize}
      \item{Name: From the well-known children's game}
    \end{itemize}
  \end{block}

  \begin{block}{Flag}
    \begin{itemize}
        \item{A hidden and / or protected file}
        \item{}
    \end{itemize}
  \end{block}
  
  \begin{block}{Challenge}
    \begin{itemize}
        \item{}
        \item{}
    \end{itemize}
  \end{block}
  

\end{frame}

%------------------------------------------------

\begin{frame}{CTF events}

  \begin{block}{Origins}
    \begin{itemize}
      \item{Name: From the well-known children's game}
    \end{itemize}
  \end{block}

  \begin{block}{Flag}
    \begin{itemize}
        \item{A hidden and / or protected file}
        \item{}
    \end{itemize}
  \end{block}
  
  \begin{block}{Challenge}
    \begin{itemize}
        \item{}
        \item{}
    \end{itemize}
  \end{block}
  

\end{frame}

%------------------------------------------------

\begin{frame}{CTF cagetories}

    \begin{itemize}
      \item{Cryptography}

      \bigskip

      \item{Web exploration}

      \bigskip

      \item{Reverse engineering}

      \bigskip

      \item{Binary exploitation}

      \bigskip

      \item{Forensics}

      \bigskip

      \item{Misc / General skills}
    \end{itemize}

\end{frame}

%------------------------------------------------

\begin{frame}{Cryptography}

  \begin{itemize}
    \item{Cryptography is used for }
    \item{Origins of cryptography go way back before IT}
    \item{Cryptography is mathematically well founded, but mis-use is easy}
    \item{Technically: Hash algorithms, key derivation standards, symmetric and public key encryptions}
    \item{Easy challenges: Caesar and monoalphabetic codes, BASE64, hexadecimal encodings}
    \item{Difficult level: Recognision of mis-uses, implementation issues. Many challenges need mathematical level understanding.}
  \end{itemize}

  \begin{exampleblock}{}
    Main benefit: mathematical level understanding of data security.
  \end{exampleblock}

\end{frame}

%------------------------------------------------

\begin{frame}{Web exploration}

  \begin{itemize}
    \item{\textbf{Web is the most ubiquitous element of the cyberworld}}
    \item{Information fragments in the HTTP, CSS or JavaScript files}
    \item{Deduction of possible unexposed files}
    \item{Cookie manipulation}
    \item{Protocol level mechanisms}
    \item{Exploitation of web languages (e.g. PHP)}
    \item{Bypassing of authitentication}
  \end{itemize}

  \begin{exampleblock}{}
    Main benefit: understanding the internals of the web technologies to develop more secure web applications
  \end{exampleblock}
\end{frame}

%------------------------------------------------

\begin{frame}{Reverse engineering}

  \begin{itemize}
    \item{}
  \end{itemize}

  \begin{exampleblock}{}
    Main benefit: 
  \end{exampleblock}

\end{frame}

%------------------------------------------------
\begin{frame}{Binary exploitation}

  \begin{itemize}
    \item{}
  \end{itemize}

  \begin{exampleblock}{}
    Main benefit: 
  \end{exampleblock}

\end{frame}

%------------------------------------------------
\begin{frame}{Forensics}

  \begin{itemize}
    \item{Digital forensics is getting more and more important, also during general investigations}
      \begin{itemize}
        \item{Was this particular file or content present on a storage
          device?}
        \item{Has a file been deleted or altered?}
        \item{What is the true creation or modification date of a file?}
        \item{Who is the genuine sender of a suspicious email?}
        \item{Was this document forged?}
      \end{itemize}
    \item{Examination of filesystem images and analyzing network traffic captures}
  \end{itemize}

  \begin{exampleblock}{}
    Main benefit: Learning important forensic techniques that can be applied during a real investigation
  \end{exampleblock}

\end{frame}

%------------------------------------------------
\begin{frame}{Misc / General skills}

  \begin{itemize}
    \item{Anything, that does not fit into the others}
    \item{Shell based skills}
    \item{Common tools (e.g. text manipulation, package managers, archivers, filesystem utils, git)}
    \item{Manipulating filesystem / virtual machine images}
    \item{New areas (timing attacks, sidechannel attacks, fault injection)}
  \end{itemize}

  \begin{exampleblock}{}
    Main benefit: To acquire skills using the widely used system tools
  \end{exampleblock}

\end{frame}

%------------------------------------------------

\end{document}

