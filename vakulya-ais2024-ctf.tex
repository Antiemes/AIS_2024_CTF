\documentclass[conference]{IEEEtran}
\IEEEoverridecommandlockouts
% The preceding line is only needed to identify funding in the first footnote. If that is unneeded, please comment it out.
\usepackage{cite}
\usepackage{amsmath,amssymb,amsfonts}
\usepackage{algorithmic}
\usepackage{graphicx}
\usepackage{textcomp}
\usepackage{xcolor}
%\usepackage{booktabs}
\usepackage{tabularray}
\usepackage{flushend}
\UseTblrLibrary{booktabs}
\def\BibTeX{{\rm B\kern-.05em{\sc i\kern-.025em b}\kern-.08em
    T\kern-.1667em\lower.7ex\hbox{E}\kern-.125emX}}
\begin{document}

\title{Gamifying Cybersecurity:\\The CTF Challenges}

\author{\IEEEauthorblockN{Gergely Vakulya}
\IEEEauthorblockA{\small \textit{Alba Regia Technical Faculty} \\
\textit{Óbuda University}\\
\textit{Székesfehérvár, Hungary}\\
\textit{vakulya.gergely@amk.uni-obuda.hu}
}
\and
\IEEEauthorblockN{Helga Anna Albert-Huszár}
\IEEEauthorblockA{\small \textit{Alba Regia Technical Faculty} \\
\textit{Óbuda University}\\
\textit{Székesfehérvár, Hungary}\\
\textit{albert.huszar.helga@amk.uni-obuda.hu}
}
}

\maketitle

\begin{abstract}
Cybersecurity is becoming increasingly critical in today's digital landscape,
  yet it remains challenging to enter the field due to its inherently complex
  nature, requiring expertise across a wide range of specialized areas. This
  paper explores the use of Capture the Flag (CTF) challenges as a gamified
  method for enhancing cybersecurity education and skill development. By
  integrating real-world scenarios into a competitive format, CTF challenges
  encourage participants to apply their knowledge in dynamic, problem-solving
  environments. We analyze various CTF challenge categories, and discuss their
  effectiveness in fostering critical cybersecurity skills.
\end{abstract}

\begin{IEEEkeywords}
cybersecurity, linux, forensics, cryptography
\end{IEEEkeywords}

\section{Introduction}

The traditional game of capture the flag involves two teams, each tasked with
defending their own flag while attempting to capture the opposing team’s flag.
In CTF challenges, however, the 'flag' typically takes the form of a hidden or
encrypted string (such as a password), which is securely protected.
Successfully capturing the flag serves as proof that the player has overcome
the challenge by breaching the system's defenses or solving the puzzle.

CTF challenges are based in information technology, often demanding a deep
understanding of various technologies that extend well beyond traditional
software development.

CTF challenges are typically organized into distinct categories, with the most
common being binary exploitation, reverse engineering, cryptography, web
technologies, and digital forensics. Each category focuses on a specialized
area, requiring participants to apply specific skills and knowledge to solve
the associated problems. There are challenges that demand expertise across multiple
fields, while others don't fit into any of the afore mentioned categories.

CTF problems are not only good for practicing, but they are competitive

time, leaderboard, utólag elérhető.

Vannak online es jelenleti esemenyek
is, illetve elerhetoek az elozo feladatok. Lehet csapatos, vagy egyeni.
A csapat osszetartozasat is jol erositi. Tobb ember tobb nezopontbol maskepp lathatja.
Writeup: Ebben egy csapat, vagy szemely megoldasa van reszletesen leirva.

\section{CTF events}

\subsection{Google CTF}

Google CTF is an annual, entirely online Capture the Flag (CTF) competition hosted by Google,
aimed at both beginner and advanced participants. The event is designed to
foster interest in cybersecurity and challenge individuals to solve a variety
of technical problems related to hacking, cryptography, and exploitation. It
has become one of the most recognized CTF events in the cybersecurity
community.

Google CTF features both a beginner-friendly and advanced division, catering to
participants with varying levels of expertise. The beginner track, sometimes
called "Beginner's Quest," allows newcomers to ease into cybersecurity by
solving easier problems. Meanwhile, the advanced track pushes experienced
hackers to their limits, tackling complex challenges across multiple
disciplines.

\subsection{PicoCTF}

picoCTF is one of the largest CTF competitions designed
primarily for middle and high school students, but it’s open to anyone
interested in learning cybersecurity. It was launched in 2013 \cite{zhang2013}
by Carnegie Mellon
University’s CyLab Security and Privacy Institute, and continues to be organized annually.

What sets picoCTF apart from other CTF competitions is its educational focus.
Unlike many CTFs, which are primarily competitive, picoCTF is designed to teach
participants fundamental security concepts while they compete. It features a
learning environment called picoGym, where users can practice their skills on
challenges from previous years at their own pace. This flexible,
learn-as-you-go format is especially useful for beginners and educators looking
to integrate cybersecurity into their curricula.

\subsection{Hungarian Cyber Security Challenge}

The Hungarian Cyber Security Challenge (HCSC) is Hungary’s premier
cybersecurity competition, organized annually by the National Cyber Security
Center. Being Open to Hungarian citizens aged 16 and above, the event is
aimed to identify and promote the next generation of
cybersecurity talent. 

Unlike most traditional CTF competitions, there are some specific challenges
where participants are required to provide detailed documentation, known as
''writeups'', explaining how they solved the problem and captured the flag. These
writeups serve as comprehensive reports that outline the methods and tools used
to exploit vulnerabilities and provide step-by-step explanations of the
problem-solving process.

Another unique aspect of this event is that the final rankings are not solely
determined by the points participants accumulate during the competition. In
addition to the scores, the organizers place significant weight on the quality
of the participants' writeups and the methods they used to solve the
challenges. This approach ensures that creativity, technical depth, and
problem-solving strategies are recognized alongside mere point accumulation.
By evaluating the techniques and documentation provided in the writeups, the
organizers reward participants who demonstrate a thorough understanding of the
challenges, innovative approaches, and detailed explanations of their
solutions.

\subsection{Riscure Hack Me CTF}

As the Internet of Things (IoT) continues to expand, it introduces significant
new security challenges, making IoT devices increasingly vulnerable to
cyberattacks. IoT ecosystems often include embedded systems with limited
security measures, creating potential entry points for malicious actors. To
highlight these vulnerabilities and raise awareness of the risks, Capture the
Flag (CTF) competitions have started incorporating hardware-based challenges.
These challenges focus on the security flaws of embedded systems, which are
critical in IoT devices \cite{prinetto2020}. %Hardware CTF
One standout example is the Riscure Hack Me (rhme) CTF, which ran three times
between 2015 and 2018.


Section \ref{sec-challenge-types} outlines the various types of CTF challenges,
emphasizing their educational benefits in the context of cybersecurity
education. This section also provides illustrative examples to demonstrate how
each challenge type contributes to skill development and knowledge
application."

% Berti cikkei, virtualis oktatas
\cite{gyorok2014}
\cite{safar2019}
\cite{beszedes2023}

%Cybersecurity oktatas
\cite{rahman2020}

%Cybersecurity az egyetemen
\cite{schneider2013}

%Buffer overflow
\cite{lhee2003}

%Steganography
\cite{morkel2005}

%Cezartol a public-keyig
\cite{luciano1987}

%Wireshark forensics
\cite{ndatinya2015}

%GDB
\cite{stallman1988}

%Ghidra
\cite{eagle2020}

%IoT vulnerabilities
\cite{butun2019}

%Fault injection
\cite{ziade2004}

\section{Challenge types}
\label{sec-challenge-types}

\begin{figure}[htbp]
	\centering
	\includegraphics[width=0.4\textwidth]{fig/gdb.png}
	\caption{Disassembling a binary function in GDB}
	\label{fig-gdb}
\end{figure}

\begin{figure}[htbp]
	\centering
	\includegraphics[width=0.5\textwidth]{fig/wireshark.png}
	\caption{Analyzing network traffic with Wireshark}
	\label{fig-wireshark}
\end{figure}

\section{Challenge categories}

CTFs are typically organized into specialized challenge types, each focusing on
specific areas of cybersecurity. This section outlines the various challenge
categories and explains how each one is useful for individuals learning
cybersecurity.

\subsection{Web exploration}

The web is one of the most ubiquitous components of the cyberworld, forming the
backbone of countless online services and interactions. Web-related challenges
in CTF competitions often leverage a wide range of technologies, including
fundamental elements such as HTML, CSS, and JavaScript, along with various
image file formats, frameworks, web servers, and communication protocols. In
the simplest challenges, participants might find hidden data embedded in the
source code, or they may need to deduce undisclosed file names that
contain sensitive information. These types of challenges encourage participants
to inspect the structure of web applications and recognize common points of
data leakage.

Other challenges explore the HTTP protocol itself, often requiring participants
to manipulate request headers, cookies, or status codes to reveal
vulnerabilities or extract data that would otherwise remain concealed. These
tasks familiarize competitors with the nuances of HTTP traffic and the
importance of understanding how data is transmitted between client and server.

A third major category involves bypassing authorization mechanisms, where
participants exploit flaws in access control or authentication processes. These
challenges often simulate real-world vulnerabilities, such as broken
authentication, insufficient role checks, or logic flaws that allow
unauthorized users to access privileged information or functionalities. By
solving these, participants gain critical skills in recognizing and preventing
access control vulnerabilities, a common issue in web application security.

\subsection{Cryptography}

\subsection{Reverse Engineering}

\subsection{Binary Exploration}

\subsection{Forensics}

\subsection{Miscellaneous}

\section{Summary}

Capture the Flag (CTF) challenges offer a highly effective way of developing
essential cybersecurity skills by providing hands-on experience in a
competitive, gamified environment. This article explores how CTFs function as
both educational tools and recruitment platforms, emphasizing their role in
making cybersecurity more accessible to beginners while offering advanced
participants opportunities to deepen their expertise. By analyzing popular CTF
events like DEF CON, picoCTF, and Google CTF, the article highlights how these
competitions engage participants with real-world cybersecurity problems, thus
fostering critical problem-solving abilities. Furthermore, it examines how CTFs
contribute to both individual skill development and the broader cybersecurity
community.

\bibliographystyle{IEEEtran}
\bibliography{IEEEabrv,references}

\end{document}
