\documentclass[conference]{IEEEtran}
\IEEEoverridecommandlockouts
% The preceding line is only needed to identify funding in the first footnote. If that is unneeded, please comment it out.
\usepackage{cite}
\usepackage{amsmath,amssymb,amsfonts}
\usepackage{algorithmic}
\usepackage{graphicx}
\usepackage{textcomp}
\usepackage{xcolor}
%\usepackage{booktabs}
\usepackage{tabularray}
\usepackage{flushend}
\UseTblrLibrary{booktabs}
\def\BibTeX{{\rm B\kern-.05em{\sc i\kern-.025em b}\kern-.08em
    T\kern-.1667em\lower.7ex\hbox{E}\kern-.125emX}}
\begin{document}

\title{Gamifying Cybersecurity:\\The CTF Challenges}

\author{\IEEEauthorblockN{Gergely Vakulya}
\IEEEauthorblockA{\small \textit{Alba Regia Technical Faculty} \\
\textit{Óbuda University}\\
\textit{Székesfehérvár, Hungary}\\
\textit{vakulya.gergely@amk.uni-obuda.hu}
}
\and
\IEEEauthorblockN{Helga Anna Albert-Huszár}
\IEEEauthorblockA{\small \textit{Alba Regia Technical Faculty} \\
\textit{Óbuda University}\\
\textit{Székesfehérvár, Hungary}\\
\textit{albert.huszar.helga@amk.uni-obuda.hu}
}
}

\maketitle

\begin{abstract}
Cybersecurity is becoming increasingly critical in today's digital landscape,
  yet it remains challenging to enter the field due to its inherently complex
  nature, requiring expertise across a wide range of specialized areas. This
  paper explores the use of Capture the Flag (CTF) challenges as a gamified
  method for enhancing cybersecurity education and skill development. By
  integrating real-world scenarios into a competitive format, CTF challenges
  encourage participants to apply their knowledge in dynamic, problem-solving
  environments. We analyze various CTF challenge categories, and discuss their
  effectiveness in fostering critical cybersecurity skills.
\end{abstract}

\begin{IEEEkeywords}
cybersecurity, linux, forensics, cryptography
\end{IEEEkeywords}

\section{Introduction}

The traditional game of capture the flag involves two teams, each tasked with
defending their own flag while attempting to capture the opposing team’s flag.
In CTF challenges, however, the 'flag' typically takes the form of a hidden or
encrypted string (such as a password), which is securely protected.
Successfully capturing the flag serves as proof that the player has overcome
the challenge by breaching the system's defenses or solving the puzzle.

CTF challenges are based in information technology, often demanding a deep
understanding of various technologies that extend well beyond traditional
software development.

CTF challenges are typically organized into distinct categories, with the most
common being binary exploitation, reverse engineering, cryptography, web
technologies, and digital forensics. Each category focuses on a specialized
area, requiring participants to apply specific skills and knowledge to solve
the associated problems. There are challenges that demand expertise across multiple
fields, while others don't fit into any of the afore mentioned categories.

CTF problems are not only good for practicing, but they are competitive

time, leaderboard, utólag elérhető.

Vannak online es jelenleti esemenyek
is, illetve elerhetoek az elozo feladatok. Lehet csapatos, vagy egyeni.
A csapat osszetartozasat is jol erositi. Tobb ember tobb nezopontbol maskepp lathatja.
Writeup: Ebben egy csapat, vagy szemely megoldasa van reszletesen leirva.

Google CTF, PicoCTF, meg az a magyar valami. Kulon irni a hardveresrol is.

Section \ref{sec-challenge-types} outlines the various types of CTF challenges,
emphasizing their educational benefits in the context of cybersecurity
education. This section also provides illustrative examples to demonstrate how
each challenge type contributes to skill development and knowledge
application."

% Berti cikkei, virtualis oktatas
\cite{gyorok2014}
\cite{safar2019}
\cite{beszedes2023}

%Cybersecurity oktatas
\cite{rahman2020}

%Cybersecurity az egyetemen
\cite{schneider2013}

%Buffer overflow
\cite{lhee2003}

%Steganography
\cite{morkel2005}

%Cezartol a public-keyig
\cite{luciano1987}

%Wireshark forensics
\cite{ndatinya2015}

%GDB
\cite{stallman1988}

%Ghidra
\cite{eagle2020}

\section{Challenge types}
\label{sec-challenge-types}

\begin{figure}[htbp]
	\centering
	\includegraphics[width=0.35\textwidth]{fig/dummy.png}
	\caption{Dummy image}
	\label{fig-dummy}
\end{figure}

\section{Challenge categories}

CTFs are typically organized into specialized challenge types, each focusing on
specific areas of cybersecurity. This section outlines the various challenge
categories and explains how each one is useful for individuals learning
cybersecurity.

\subsection{Web exploration}

\subsection{Cryptography}

\subsection{Reverse Engineering}

\subsection{Binary Exploration}

\subsection{Forensics}

\subsection{Miscellaneous}

\section{Summary}

Capture the Flag (CTF) challenges offer a highly effective way of developing
essential cybersecurity skills by providing hands-on experience in a
competitive, gamified environment. This article explores how CTFs function as
both educational tools and recruitment platforms, emphasizing their role in
making cybersecurity more accessible to beginners while offering advanced
participants opportunities to deepen their expertise. By analyzing popular CTF
events like DEF CON, picoCTF, and Google CTF, the article highlights how these
competitions engage participants with real-world cybersecurity problems, thus
fostering critical problem-solving abilities. Furthermore, it examines how CTFs
contribute to both individual skill development and the broader cybersecurity
community.

\bibliographystyle{IEEEtran}
\bibliography{IEEEabrv,references}

\end{document}
